\documentclass[12pt]{article}
\usepackage[a4paper, margin=2cm]{geometry}
\usepackage[utf8]{inputenc}
\usepackage{graphicx}
\usepackage{lastpage}
\usepackage{fancyhdr}
\usepackage{tikz}
\usetikzlibrary{trees}
\pagestyle{fancy}
\fancyhf{}
\rfoot{Page \thepage \hspace{1pt} sur \pageref{LastPage}}

\begin{document}
\section*{Le binôme de Newton}
\subsection*{Mise au point de la formule}
On peut exprimer facilement les formules donnant un développement des puissances (entières et positives) du binôme $(x+y)$ lorsque $n=1, 2, 3$ ou $4$. On a en effet:\vspace{8pt}

$(x+y)^1=1.x+1.y$\vspace{6pt}

$(x+y)^2=1.x^2+2.xy+1.y^2$\vspace{6pt}

$(x+y)^3=1.x^3+3.x^2y+3.xy^2+1.y^3$\vspace{6pt}

$(x+y)^4=1.x^4+4.x^3y+6.x^2y^2+4.xy^3+1.y^4$\vspace{2pt}

\begin{flushleft}On constate que les coefficients apparaissent dans le triangle de Pascal. On peut ainsi écrire :\end{flushleft}

$(x+y)^1=C_1^0.x+C_1^1.y$\vspace{6pt}

$(x+y)^2=C_2^0.x^2+C_2^1.xy+C_2^2.y^2$\vspace{6pt}

$(x+y)^3=C_3^0.x^3+C_3^1.x^2y+C_3^2.xy^2+C_3^3.y^3$\vspace{6pt}

$(x+y)^4=C_4^0.x^4+C_4^1.x^3y+C_4^2.x^2y^2+C_4^3.xy^3+C_4^4.y^4$

\subsection*{Généralisation}


\begin{eqnarray*}
(x+y)^n&=&C_n^0.x^n+C_n^1.x^{n-1}y+C_n^2.x^{n-2}y^2+...+C_n^i.x^{n-i}y^i+...+C_n^n.y^n\\
&=&\sum_{i=0}^n C_n^ix^{n-i}y^i\\
&=&(y+x)^n\\
&=&\sum_{i=0}^n C_n^ix^iy^{n-i}\\
\end{eqnarray*}

\subsection*{Corollaire}

\begin{eqnarray*}
(x+y)^n&=&(x+(-y))^n\\
&=&C_n^0.x^n-C_n^1.x^{n-1}y+C_n^2.x^{n-2}y^2+...+(-1)^iC_n^i.x^{n-i}y^i+...+(-1)^nC_n^n.y^n\\
&=&\sum_{i=0}^n(-1)^iC_n^ix^{n-i}y^i\\
\end{eqnarray*}

\subsection*{Tableau des valeurs $C_n^p$ avec $n\geq p$}

\begin{tabular}{|c||c|c|c|c|c|c|c|c|}
\hline
$C_n^p$ & 0 & 1 & 2 & 3 & 4 & 5 & 6 & 7\\ \hline \hline
0 & 1 & & & & & & &\\ \hline
1 & 1 & 1 & & & & & &\\ \hline
2 & 1 & 2 & 1 & & & & &\\ \hline
3 & 1 & 3 & 3 & 1 & & & &\\ \hline
4 & 1 & 4 & 6 & 4 & 1 & & &\\ \hline
5 & 1 & 5 & 10 & 10 & 5 & 1 & &\\ \hline
6 & 1 & 6 & 15 & 20 & 15 & 6 & 1 &\\ \hline
7 & 1 & 7 & 21 & 35 & 35 & 21 & 7 & 1\\ \hline
\end{tabular}

\end{document}