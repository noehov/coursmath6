\documentclass[10pt]{article}
\usepackage[a4paper, margin=2cm]{geometry}
\usepackage[utf8]{inputenc}
\usepackage{graphicx}
\usepackage{lastpage}
\usepackage{fancyhdr}
\usepackage{tikz}
\usetikzlibrary{trees}
\pagestyle{fancy}
\fancyhf{}
\rfoot{Page \thepage \hspace{1pt} sur \pageref{LastPage}}
\begin{document}
\title{Math : Probabilités - Partie IV : Variables aléatoires}
\author{Hovinne Noé}
\date{13.06.20}
\maketitle
\section*{Exercice 1, page 5}
On tire au hasard une carte d’un jeu de 32 cartes. Soit X la variable aléatoire qui
associe à une carte, la valeur 4 si c’est un as, 3 pour un roi, 2 pour une dame, 1 pour
un valet et 0 pour le reste.

$$P(X=4)=\frac{4}{32}=\frac{1}{8}$$
$$P(X=3)=\frac{4}{32}=\frac{1}{8}$$
$$P(X=2)=\frac{4}{32}=\frac{1}{8}$$
$$P(X=1)=\frac{4}{32}=\frac{1}{8}$$
$$P(X=0)=\frac{16}{32}=\frac{1}{2}$$


$$E(X)=4\times\frac{1}{8}+3\times\frac{1}{8}+2\times\frac{1}{8}+1\times\frac{1}{8}=\frac{\textbf{5}}{\textbf{4}}$$

$$V(X)=\left[4^2\times\frac{1}{8}+3^2\times\frac{1}{8}+2^2\times\frac{1}{8}+1^2\times\frac{1}{8}\right]-E^2(X)=\frac{30}{8}-\frac{25}{16}=\frac{\textbf{35}}{\textbf{16}}$$

$$\sigma(X)=\sqrt{\frac{35}{16}}=\frac{\sqrt{\textbf{35}}}{\textbf{4}}$$
\section*{Exercice 2, page 6}
On jette deux dés bien équilibrés. X désigne la somme des points obtenus.

$$P(X=12)=p_1=\frac{1}{36}$$
$$P(X=11)=p_2=\frac{2}{36}$$
$$P(X=10)=p_3=\frac{3}{36}$$
$$P(X=9)=p_4=\frac{4}{36}$$
$$P(X=8)=p_5=\frac{5}{36}$$
$$P(X=7)=p_6=\frac{6}{36}$$
$$P(X=6)=p_7=\frac{5}{36}$$
$$P(X=5)=p_8=\frac{4}{36}$$
$$P(X=4)=p_9=\frac{3}{36}$$

\newpage
$$P(X=3)=p_{10}=\frac{2}{36}$$
$$P(X=2)=p_{11}=\frac{1}{36}$$
$$E(X)=\sum_{i=1}^{11}x_i.p_i=\textbf{7}$$
$$V(X)=\sum_{i=1}^{11}(x_i^2.p_i)-E^2(X)=\frac{\textbf{35}}{\textbf{6}}$$
$$\sigma(X)=\frac{\sqrt{\textbf{210}}}{\textbf{6}}$$
\section*{Exercice 3, page 6}
On tire au hasard 1 échantillon de 3 articles dans une boîte de 12 articles dont 3 sont défectueux. X est la variable aléatoire qui s'intéresse au nombre de pièces défectueuses.

$$P(X=1)=p_1=C_3^1\times\frac{3.9.8}{12.11.10}=\frac{27}{55}$$
$$P(X=2)=p_2=C_3^2\times\frac{3.2.9}{12.11.10}=\frac{27}{220}$$
$$P(X=3)=p_3=C_3^3\times\frac{3.2.1}{12.11.10}=\frac{6}{1320}$$
$$P(X=0)=\frac{9.8.7}{12.11.10}=\frac{21}{55}$$
$$E(X)=\sum_{i=1}^3x_i.p_i=\frac{\textbf{3}}{\textbf{4}}$$
$$V(X)=\sum_{i=1}^3(x_i^2.p_i)-E^2(X)=\frac{\textbf{81}}{\textbf{176}}$$
$$\sigma(X)=\frac{\textbf{9}\sqrt{\textbf{11}}}{\textbf{44}}$$
\section*{Exercice 4, page 6}
Un groupe comprend 5 garçons et 3 filles. On choisit 4 personnes au hasard. X
désigne le nombre de filles.

$$P(X=1)=p_1=C_4^1\times\frac{3.5.4.3}{8.7.6.5}=\frac{3}{7}$$
$$P(X=2)=p_2=C_4^2\times\frac{3.2.5.4}{8.7.6.5}=\frac{3}{7}$$
$$P(X=3)=p_3=C_4^3\times\frac{3.2.1.5}{8.7.6.5}=\frac{1}{14}$$
$$P(X=0)=\frac{5.4.3.2}{8.7.6.5}=\frac{1}{14}$$
$$E(X)=\sum_{i=1}^3x_i.p_i=\frac{\textbf{3}}{\textbf{2}}$$
$$V(X)=\sum_{i=1}^3(x_i^2.p_i)-E^2(X)=\frac{\textbf{15}}{\textbf{28}}$$
$$\sigma(X)=\frac{\sqrt{\textbf{105}}}{\textbf{14}}$$

\newpage
\section*{Exercice 1, page 10}
On considère les familles de 10 enfants. On suppose que la probabilité d'avoir une
fille ou un garçon est identique.

a. La probabilité qu'une famille comporte deux filles est égale à :

$$P(X=2)=p_2=C_{10}^2.\left(\frac{1}{2}\right)^{10}=\frac{45}{1024}\approx\textbf{4,39\%}$$

b. La probabilité qu'une famille ne comporte pas plus de deux garçons est égale à :

$$P(X=2)+P(X=1)+P(X=0)=\frac{45}{1024}+\left[C_{10}^1.\left(\frac{1}{2}\right)^{10}\right]+\left[C_{10}^0.\left(\frac{1}{2}\right)^{10}\right]=\frac{7}{128}\approx\textbf{5,47\%}$$

c. La probabilité qu'une famille comporte au minimum deux filles est égale à :

$$\sum_{i=2}^{10}C_{10}^i.0,5^{10}=\frac{1013}{1024}\approx\textbf{98,93\%}$$
\section*{Exercice 2, page 10}
Une urne contient 10 boules blanches et 5 boules noires indiscernables au toucher.
On effectue 20 tirages avec remise de la boule dans l’urne après avoir noté sa
couleur.

a. La probabilité d'obtenir 7 boules blanches et 13 boules noires est égale à :

$$C_{20}^7.\left(\frac{10}{15}\right)^7.\left(\frac{5}{15}\right)^{13}\approx\textbf{0,2858\%}$$

b. La probabilité d'obtenir 13 boules blanches et 7 boules noires est égale à :

$$C_{20}^{13}.\left(\frac{10}{15}\right)^{13}.\left(\frac{5}{15}\right)^7\approx\textbf{18,21\%}$$
\section*{Exercice 3, page 11}
On sait qu'il y aura 40\% d'échec, à la première session.

a. En prenant cinq candidats au hasard, la probabilité qu'aucun des cinq ne réussisse est égale à :

$$C_5^0.0,4^5=\textbf{1,024\%}$$

b. En prenant cinq candidats au hasard, la probabilité que les cinq réussissent est égale à :

$$C_5^5.0,6^5=\textbf{7,776\%}$$

c. En prenant cinq candidats au hasard, la probabilité qu'au moins deux réussissent est égale à :

$$\sum_{i=2}^5C_5^i.0.6^i.0,4^{5-i}\approx\textbf{91,296\%}$$
\section*{Exercice 4, page 11}
Un caractère héréditaire a 58\% de chances d'être transmis des parents aux enfants. Considérons les familles de 4 enfants.

a. La probabilité que ce caractère se manifeste chez tous les enfants est égale à :

$$C_4^4.0,58^4\approx\textbf{11,32\%}$$

b. La probabilité que ce caractère ne se manifeste chez aucun des enfants est égale à :

$$C_4^0.0,42^4\approx\textbf{3,11\%}$$
\newpage
c. La probabilité que ce caractère ne se manifeste au plus chez trois d'entre eux est égale à :

$$\sum_{i=0}^3C_4^i.0,58^i.0,42^{4-i}\approx\textbf{88,68\%}$$

d. La probabilité que ce caractère se manifeste au moins chez deux d'entre eux est égale à :

$$\sum_{i=2}^4C_4^i.0,58^i.0,42^{4-i}\approx\textbf{79,70\%}$$
\section*{Exercice 5, page 11}
La probabilité qu'un tireur atteigne une cible est de 0,25.

a. En supposant qu'il tire 7 fois, la probabilité qu'il atteigne la cible au moins 2 fois est égale à :

$$\sum_{i=2}^7C_7^i.0,25^i.0,75^{7-i}\approx\textbf{55,47\%}$$

b. Pour que la probabilité d'atteindre la cible au moins une fois soit supérieure ou égale à $2/3$, quel est le 

nombre de tirs que le tireur doit effectuer ?\vspace{2pt}

$$P(X\geq1)\geq\frac{2}{3}\Leftrightarrow P(X=0)\leq\frac{1}{3}$$
\begin{center} Or, on sait que
$$P(X=0)=C_n^0.0,25^0.0,75^{n-0}$$
Donc,
$$C_n^0.0,25^0.0,75^{n-0}\leq\frac{1}{3}$$
$$0,75^n\leq\frac{1}{3}\Leftrightarrow n\geq\textbf{4}$$
Le nombre de tirs à réaliser est égal à \textbf{4}.\end{center}
\section*{Exercice 1, page 25}
La moyenne des capacités respiratoires d'un échantillon de 400 personnes du sexe masculin est de 3,7 L avec un écart-type de 0,7 L.

Sachant que les capacités respiratoires sont distribuées suivant une loi normale, le nombre de personnes 

ayant une capacité respiratoire comprise entre 3 L et 4,4 L que l'on peut s'attendre à trouver est égal à :

$$P(3\leq X\leq 4,4)=0,68268949\approx 68,27\%$$

$$0,68268949\times400=273,075796\approx \textbf{273}$$

\section*{Exercice 2, page 25}
Une étude faite par un fabricant de lampes électriques a montré que les durées des lampes se répartissent approximativement suivant une loi de Laplace-Gauss, la moyenne de vie étant de 1200 heures et l’écart-type étant de 180 heures. On considère une installation de 10 000 lampes neuves.

a. Le nombre d'ampoules que l'on peut prévoir hors d'usage au bout de 1000 heures est égal à :

$$P(X\leq1000)=P(-\infty\leq X\leq1000)=0,13326026\approx13,33\%$$

$$0,13326026\times10000=1332,6026\approx\textbf{1333}$$
\newpage

b. Le nombre d'ampoules que l'on peut prévoir hors d'usage entre 1000 et 1500 heures d'utilisation est égal 

à :

$$P(1000\leq X\leq1500)=0,81894938\approx81,89\%$$

$$0,81894938\times10000=8189,4938\approx\textbf{8189}$$
\vspace{21cm}
\begin{center}
\vspace{1cm}Ce travail a été réalisé en \LaTeX . \vspace{0.2cm}
\end{center}
\end{document}
